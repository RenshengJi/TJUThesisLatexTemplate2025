\chapter{理论研究}
此处格式已按模板设定,作者只需选择段落区域,输入替换之。模版中所有说明性文字用于注释格式与内容的要求,撰写论文时请删除。模版中,图表、公式、参考文献等都已给出范例,撰写论文时请删除。

本模版已包含符合章节设置的“多级别列表”,只需在相应位置替换标题文字即可。如需增加章节,建议先使用格式刷,再调整编号。

\section{结构及要求}
\subsection{论文结构}
毕业设计(论文)原则上应采用中文完成(教学语言为英语的除外),确需用其他语言撰写的需经学院审批同意并报教务处备案。毕业设计(论文)一般由以下部分组成,依次为:①封面;②扉页;③独创性声明;④中英文摘要及关键词;⑤目录;⑥正文;⑦参考文献;⑧附录;⑨致谢。

\subsection{语言表述}
要做到数据可靠、推理严谨、立论正确。论述必须简明扼要、重点突出,对同行专业人员已熟知的常识性内容,尽量减少叙述。

论文中如出现一些非通用性的新名词、术语或概念,需做出解释。

\subsection{标题和层次}
标题要重点突出,简明扼要,层次要清楚。

\subsection{打印规格}
论文一律采用A4纸张(大小为210mm×297mm)打印,可根据实际选择单面或双面印刷,页边距如下设置。上:27.5mm;下:25.4mm;左:35.7mm;右:27.7mm。页眉距边界15.0mm;页脚距边界17.5mm。字符间距为默认值(缩放100%,间距:标准)。

\section{论文撰写规范}

\subsection{封面}
采用天津大学本科生毕业设计(论文)统一封面,封面内容包括论文题目、学院、专业、年级、姓名、学号、指导教师等信息。

论文题目是论文总体内容的体现,要求醒目、简明、准确,主题突出,一般不宜超过25字。

\subsection{目录}
目录的各章节应简明扼要,应列至三级标题,包含正文及其后的各部分,并附有相应页码。目录的文字应与相应标题文字完全一致。

“目录”两字之间空一个全角空格或两个半角空格,采用不编号章标题样式。目录条目采用正文样式。

各级标题采用逐级缩进形式,每级缩进2字符,页码前导符采用“…”。

\subsection{正文}
正文是毕业设计(论文)的主体,应占据主要篇幅,文字一般不少于15000字,要求主题明确,内容充实;论点正确,论据可靠,论证充分。内容一般包括:设计(论文)的工作目的(背景),国内外研究现状、理论分析、计算方法、实验装置和测试方法、实验结果分析与讨论、研究成果、结论及意义等。

正文中文字体为宋体,英文字体为Times New Roman,正文采用小四号字,段落行间距为固定值20磅,段落前后间距为0,首行缩进2字符。西文字体以Times New Roman为准,若Times New Roman中没有相应字符,则应使用较为清晰和通用的字体。数学公式和专门文字(如计算机程序代码)的字体可以根据需要选择。

章节与标号:一般分为章标题(一级标题)、不编号章标题(同属于一级标题)、二级标题和三级标题。各章节编号建议采用Word的“多级别列表”方式自动形成编号,标题编号与标题内容之间空一个全角空格或两个半角空格。各级章节标题格式要求细节参见《天津大学本科生毕业设计(论文)撰写规范》。

图、表等与其前后的正文之间要有一行的间距;文中的图、表、公式一律采用阿拉伯数字分章编号,如:图2-5,表3-2,公式(5-1)(“公式”两个字不要写上)等。若图或表中有附注,采用英文小写字母顺序编号。子图采用英文字母编号。引用图或表应在图题或表题右上角标出文献来源。图或表的附注应位于图或表的下方。
